% Ubah kode buku berikut dengan yang ditentukan oleh departemen
\begin{large}
  FINAL PROJECT PROPOSAL - EC234701
\end{large}

\vspace{\fill}

% Ubah kalimat berikut dengan judul tugas akhir
\begin{spacing}{1.5}
  \begin{Large}
    IOT BASED ELECTRICITY TOKEN FILLING TOOL USING 2-AXIS GANTRY SYSTEM
  \end{Large}
\end{spacing}

\vspace{\fill}

% Ubah kalimat-kalimat berikut dengan nama dan NRP mahasiswa
\begin{large}
  Rayhan Narindran Cendikia \\
  \textmd{NRP 5024211022}
\end{large}

\vspace{\fill}

% Ubah kalimat-kalimat berikut dengan nama-nama dosen pembimbing
\begin{large}
  \textmd{Advisor} \\
  Arta Kusuma Hernanda, S.T., M.T. \\
  \textmd{NPP 1996202311024} \\
  Dr. Arief Kurniawan, S.T., M.T. \\
  \textmd{NIP 197409072002121001}
\end{large}

\vspace{\fill}

% Ubah kalimat-kalimat berikut dengan nama departemen dan fakultas
Undergraduate Study Program of Computer Engineering \\

\mdseries

Department of Computer Engineering \\
Faculty of Intelligent Electrical and Informatics Technology \\
Sepuluh Nopember Institute of Technology

% Ubah kalimat berikut dengan tempat dan tahun pembuatan buku
Surabaya \\
2024


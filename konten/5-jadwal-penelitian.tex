\chapter{JADWAL PENELITIAN}

% Ubah tabel berikut sesuai dengan isi dari rencana kerja
\newcommand{\w}{}
\newcommand{\G}{\cellcolor{gray}}
\begin{table}[H]
  \captionof{table}{Tabel timeline}
  \label{tbl:timeline}
  \begin{tabular}{|p{3.5cm}|c|c|c|c|c|c|c|c|c|c|c|c|c|c|c|c|}

    \hline
    \multirow{2}{*}{Kegiatan} & \multicolumn{16}{|c|}{Minggu}                                                                       \\
    \cline{2-17}              &
    1                         & 2                             & 3  & 4  & 5  & 6  & 7  & 8  & 9  & 10 & 11 & 12 & 13 & 14 & 15 & 16 \\
    \hline

    % Gunakan \G untuk mengisi sel dan \w untuk mengosongkan sel
    Perancangan desain alat          &
    \G                        & \G                            & \w & \w & \w & \w & \w & \w & \w & \w & \w & \w & \w & \w & \w & \w \\
    \hline

    Implementasi hardware           &
    \w                        & \w                            & \G & \G & \G & \G & \G & \G & \w & \w & \w & \w & \w & \w & \w & \w \\
    \hline

    Pemrograman alat             &
    \w                        & \w                            & \w & \w & \w & \w & \w & \w & \G & \G & \G & \G & \G & \G & \w & \w \\
    \hline

    Evaluasi alat       &
    \w                        & \w                            & \w & \w & \w & \w & \w & \w & \w & \w & \w & \w & \G & \G & \G & \G \\
    \hline

    Penulisan buku           &
    \w                        & \w                            & \w & \w & \w & \w & \w & \w & \w & \w & \G & \G & \G & \G & \G & \G \\
    \hline
  \end{tabular}
\end{table}

Pada \emph{timeline} yang tertera di Tabel \ref{tbl:timeline} di atas, terdapat 4 kegiatan komponen utama yang akan dilakukan dalam penelitian ini, yaitu:
\begin{enumerate}
  \item Perancangan desain alat\\
  Perancangan alat dilakukan pada minggu pertama hingga minggu kedua, dimana saat ini perancangan alat sudah berjalan sehingga hanya membutuhkan
  sedikit perubahan pada desain alat yang sudah ada.
  \item Implementasi hardware\\
  Untuk implementasi hardware, dilakukan pada minggu ketiga hingga minggu delapan, dimana saat ini alat sudah bisa bergerak secara vertikal dan sudah bisa dikontrol,
  selanjutnya akan dilakukan implementasi hardware untuk bagian yang bergerak secara horizontal dan bagian yang akan menekan tombol pada meteran listrik prabayar.
  Tahap ini membutuhkan waktu yang panjang untuk memastikan alat dapat bekerja dengan baik juga digunakan waktu tersebut
  untuk \textit{troubleshooting} dan menyesuaikan desain sesuai kebutuhan dari hardware yang digunakan
  \item Pemrograman alat\\
  Pemrograman alat dilakukan pada minggu kesembilan hingga minggu ke-14 dimana saat ini alat
  tahap pemrograman sampai pada tahap perputaran stepper motor secara vertikal yang diprogram
  menggunakan Arduino IDE dan library AccelStepperMotor. Selanjutnya program akan dikembangkan sehingga pergerakan yang dilakukan
  dapat dikontrol dengan lebih presisi juga dengan bantuan limit switch untuk kalibrasi dan melakukan limit pada pergerakan vertikal.
  Kemudian dilanjutkan dengan pemrograman terhadap pergerakan horizontal dan penekanan tombol pada meteran listrik prabayar.
  \item Evaluasi alat\\
  Tahap terakhir adalah evaluasi dari alat dimana memastikan program dan hardware dapat bekerja satu sama lain
  sesuai dengan harapan keluaran penelitian, dimana tahap ini menjadi tahap iteratif yang akan dilakukan hingga alat dapat bekerja dengan baik.
\end{enumerate}

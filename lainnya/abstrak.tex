\chapter*{ABSTRAK}
\begin{center}
  \large
  \textbf{ALAT PENGISI TOKEN LISTRIK MENGGUNAKAN SISTEM \emph{2-AXIS GANTRY} BERBASIS IOT}
\end{center}
\addcontentsline{toc}{chapter}{ABSTRAK}
% Menyembunyikan nomor halaman
\thispagestyle{empty}

\begin{flushleft}
  \setlength{\tabcolsep}{0pt}
  \bfseries
  \begin{tabular}{ll@{\hspace{6pt}}l}
  Nama Mahasiswa / NRP&:& Rayhan Narindran Cendikia / 5024211022\\
  Departemen&:& Teknik Komputer ITS\\
  Dosen Pembimbing&:& 1. Arta Kusuma Hernanda, S.T., M.T.\\
  & & 2. Dr. Arief Kurniawan, S.T., M.T\\
  \end{tabular}
  \vspace{4ex}
\end{flushleft}
\textbf{Abstrak}

% Isi Abstrak
Alat Pengisi Token Listrik Menggunakan Sistem \textit{2-Axis Gantry} berbasis IoT merupakan sebuah alat
yang dapat digunakan untuk pengisian token listrik dari jarak jauh, bekerja menggunakan sistem gantry 
dan alat IoT untuk mengontrol dan menggunakan alat dari jauh, pengguna cukup memasukkan angka 
token listrik ke aplikasi web-view kemudian alat otomatis akan menekan tombol sesuai dengan token 
yang diberikan. Alat ini juga mempermudah pengguna untuk memonitoring pulsa pada meteran listrik 
dengan adanya kamera dan mikrofon yang dapat mengingatkan pengguna jika pulsa listrik sudah mulai habis.

Alat ini dibuat dengan konsep IoT atau Internet of Things yang memungkinkan pengguna untuk melakukan kontrol
terhadap sebuah perangkat elektronika dari jarak jauh. Kontrol sendiri dibantu dengan mikrokontroler
ESP32 yang melakukan kontrol terhadap sistem \textit{2-Axis Gantry}, baik pergerakan vertikal, horizontal,
maupun aksi penekanan tombol. Untuk kerangka dari alat ini dihasilkan dengan teknologi 3D Printing, yang
dapat mempermudah pembuatan protitipe juga perubahan desain yang cepat.

Proses realisasi dari alat ini dimulai dengan melakukan perancangan prototipe alat menggunakan software CAD,
kemudian dilanjutkan dengan pembuatan rangka alat menggunakan 3D Printing. Setelah rangka alat selesai,
dilakukan implementasi perangkat elektronika dan perangkat IoT. Setelah semua perangkat terintegrasi,
alat siap diprogram sesuai dengan kebutuhan pengisian listrik dan memastikan dapat melaksanakan kontrol
dengan baik, kemudian alat siap masuk ke tahap uji coba iteratif dan evaluasi pada lapangan.

\vspace{2ex}
\noindent
\textbf{Kata Kunci: \emph{Alat, Token Listrik, Komunikasi Wireless, IoT, 3D Printing, ESP32}}
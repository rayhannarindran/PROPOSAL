\chapter*{ABSTRAK}
\begin{center}
  \large
  \textbf{ALAT PENGISI TOKEN LISTRIK MENGGUNAKAN SISTEM \emph{2-AXIS GANTRY} BERBASIS IOT}
\end{center}
\addcontentsline{toc}{chapter}{ABSTRAK}
% Menyembunyikan nomor halaman
\thispagestyle{empty}

\begin{flushleft}
  \setlength{\tabcolsep}{0pt}
  \bfseries
  \begin{tabular}{ll@{\hspace{6pt}}l}
  Nama Mahasiswa / NRP&:& Rayhan Narindran Cendikia / 5024211022\\
  Departemen&:& Teknik Komputer ITS\\
  Dosen Pembimbing&:& 1. Arta Kusuma Hernanda, S.T., M.T.\\
  & & 2. PEMBIMBING 2\\
  \end{tabular}
  \vspace{4ex}
\end{flushleft}
\textbf{Abstrak}

% Isi Abstrak
Alat Pengisi Token Listrik Menggunakan Sistem \textit{2-Axis Gantry} berbasis IoT merupakan sebuah alat
yang dapat digunakan untuk pengisian token listrik dari jarak jauh, bekerja menggunakan sistem gantry 
dan alat IoT untuk mengontrol dan menggunakan alat dari jauh, pengguna cukup memasukkan angka 
token listrik ke aplikasi web-view kemudian alat otomatis akan menekan tombol sesuai dengan token 
yang diberikan. Alat ini juga mempermudah pengguna untuk memonitoring pulsa pada meteran listrik 
dengan adanya kamera dan mikrofon yang dapat mengingatkan pengguna jika pulsa listrik sudah mulai habis.

\vspace{2ex}
\noindent
\textbf{Kata Kunci: \emph{Alat, Token Listrik, Jarak Jauh}}
\chapter*{ABSTRACT}
\begin{center}
  \large
  \textbf{IOT BASED ELECTRICITY TOKEN FILLING TOOL USING 2-AXIS GANTRY SYSTEM}
\end{center}
% Menyembunyikan nomor halaman
\thispagestyle{empty}

\begin{flushleft}
  \setlength{\tabcolsep}{0pt}
  \bfseries
  \begin{tabular}{lc@{\hspace{6pt}}l}
  Student Name / NRP&: &Rayhan Narindran Cendikia / 5024211022\\
  Department&: &Computer Engineering ITS\\
  Advisor&: &1. Arta Kusuma Hernanda, S.T., M.T.\\
  & & 2. Dr. Arief Kurniawan, S.T., M.T\\
  \end{tabular}
  \vspace{4ex}
\end{flushleft}
\textbf{Abstract}

% Isi Abstrak
IoT Based Electricity Token Filling Tool Using 2-Axis Gantry System is a tool that can be used for 
charging electricity tokens remotely, working using a gantry system and IoT tools to control and use 
tools from afar, users simply enter the electricity token number into the web-view application then 
the tool will automatically press the button according to the token given. This tool also makes it 
easier for users to monitor the credit on the electricity meter with a camera and microphone that can 
remind users if the electricity credit has started to run out.

This tool is made with the concept of IoT or Internet of Things which allows users to control an electronic device remotely.
an electronic device remotely. The control itself is assisted by a microcontroller
ESP32 microcontroller that controls the \textit{2-Axis Gantry} system, both vertical, horizontal movements,
and keystroke action. For the framework of this tool is produced with 3D Printing technology, which can facilitate the making of prototypes as well as changes to the framework.
can facilitate prototyping as well as rapid design changes.

The realization process of this tool starts with designing the tool prototype using CAD software,
then proceed with making the tool frame using 3D Printing. After the tool frame is complete,
implementation of electronic devices and IoT devices. After all devices are integrated,
device is ready to be programmed according to the needs of electricity charging and ensure that it can carry out control well.
Then the tool is ready to enter the iterative trial and evaluation stage in the field.

\vspace{2ex}
\noindent
\textbf{Keywords: \emph{Tool, Electrical Token, Wireless Communication, IoT, 3D Printing, ESP32}}